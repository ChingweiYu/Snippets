\chapter{Problems on graphs}
\begin{refsection}

\section{Tree}
\begin{Exercise}
Let $T$ be an $n$-node tree rooted at some node $r$. We want to place as few guards as possible on nodes in $T$, such that every edge of $T$ is guarded: an edge between a parent node $v$ and its child $w$ is guarded if one places a guard on at least one of these two nodes $v$, $w$. Give an $O(n)$ time algorithm for finding an optimal solution to the problem. Please show the analysis on the time and correctness of your algorithm.\school{[NTUT CSIE 101]}
\end{Exercise}
\begin{Answer}
\end{Answer}

\subsection{Lowest common ancestor}
\begin{Exercise}
\begin{enumerate}
\item Let $T$ be a binary tree rooted at $r$ with vertex set $V$ and edge set $E$. Suppose it is represented using adjacency list format. If node $u$ is an ancestor of $v$, there is a path from $r$ to $v$ passing through $u$. Consider the function $\text{ancestor}(u, v)$ which returns TRUE if $u$ is a ancestor of $v$ and FALSE otherwise. In order to have this function run in $O(1)$ time, we are asked to design an algorithm to preprocess the tree. Please provide a linear-time, i.e., $O(|V| + |E|)$ time algorithm for this preprocess. \school{[NTUT CSIE 100]}
\item \begin{enumerate}
\item Let $T$ be a binary search tree, where each vertex contains a pointer to its parent and pointers to its children, and also a field named temp, which is of type integer. You are given two pointers $q_1$, $q_2$, pointing to two vertices $v_1$, $v_2$ in $T$. Find in time $O(k)$ what is the length of the shortest path in T connecting $v_1$ to $v_2$, where $k$ is the length of this path. You may assume that before the execution of your program, the value of all the "temp" fields is zero, and you can use these fields for your algorithm.
\item Same as above, but this time the "temp" fields do not exist, you cannot write on the tree (so its information is "read-only" and the expected running time of your algorithm should be $O(k)$.
Hint: Assume that T is stored in the memory of your computer and you can find, in time $O(1)$, what is the address in which each node is stored. \school{[CCU CSIE 93]}
\end{enumerate}
\end{enumerate}
\end{Exercise}
\begin{Answer}
\end{Answer}

\section{Traversal}
\subsection{DFS}
\begin{Exercise}
\begin{enumerate}
\item An undirected graph $G = (V, E)$ is stored in a text file with the following format: The first line contains two integer numbers $n$ and $m$ that denote the numbers of vertices and edges of $G$ respectively. Then, the first line is followed by $m$ lines. Each line contains two distinct integers, say $i$ and $j$, indicating that there is an edge between vertices $i$ and $j$. Given such a file, design an $O(n)$ time algorithm to test if the undirected graph represented by the file is a tree. You should specify the data structure used to store the graph in the memory and how you construct such a data structure.  \school{[NCU CSIE 96, NTU CSIE 99]}
\item For an undirected graph $G = (V, E)$, a vertex $v \in V$, and an edge $(x, y) \in E$, let $G\setminus v$ denote the subgraph of $G$ obtained by removing $v$ and all the edges incident to $v$ from $G$; and let $G\setminus(x, y)$ denote the subgraph of $G$ obtained by removing the edge $(x, y)$ from $G$. If $G$ is connected, then $G\setminus v$ can be disconnected or connected.
\begin{enumerate}
\item Given a connected graph $G$, design an $O(|V|)$ time algorithm to find a vertex $v \in G$ such that $G\setminus v$ is connected.
\item Given a connected graph $G$, design an $O(|V|)$ time algorithm to either find an edge $(x, y) \in G$ such that $G\setminus (x, y)$  is connected or report that no such an edge exists. \school{[NCU CSIE 102]}
\end{enumerate}
\end{enumerate}
\end{Exercise}
\begin{Answer}
\end{Answer}

\subsection{BFS}

\begin{Exercise}
\begin{enumerate}
\item Given is a directed graph $G = (V, E)$ represented via adjacency lists and a vertex $v_a \in V$. Design an algorithm that outputs the length of the shortest cycle containing $v_a$ in $G$. your algorithm should solve the problem in $O(|V| + |E|)$ time.\school{[NTHU CSIE 95]}
\item We have a directed graph $G = (V, E)$ represented using adjacency lists. The edge costs are integers in the range $\{1, 2, 3, 4, 5\}$. Assume that $G$ has no self-loops or multiple edges. Design an algorithm that solves the single-source shortest path problem on $G$ in $O(|V|+|E|)$.\school{[NTHU CSIE 95]}
\end{enumerate}
\end{Exercise}
\begin{Answer}
\end{Answer}

\subsection{Topological sort}
\begin{Exercise}
Professor Lee wants to construct the tallest tower possible out of building blocks. She has n types of blocks, and an unlimited supply of blocks of each type. Each type-$i$ block is rectangular solid with linear dimension $(x_i, y_i, z_i)$. A block can be oriented so that any two of its three dimensions determine the dimensions of a base and the other dimension is the height. In building a tower, one block may be placed on top of another block as long as the two dimensions of the lower block. (Thus, for example, blocks oriented to have equal-sized bases cannot be stacked.) Use graph model to design an efficient algorithm to determine the tallest tower that the professor can build. Analyze the run time complexity. \school{[CYCU CSIE 92]}
\end{Exercise}
\begin{Answer}
\end{Answer}

\section{Path}
\begin{Exercise}
Given a graph $G = (V, E)$ and a weight function $w: E \rightarrow R$, describe a method to decide whether there is a function $h: V \rightarrow R$ such that the new weight function $w_h$ defined b $w_h = w(u, v) + h(u) - h(v)$ is non-negative. \school{[NTPU CSIE 100]}
\end{Exercise}
\begin{Answer}
\end{Answer}

\begin{Exercise}
Given an $N$ by $N$ positive matrix $R$ (i.e., each entry $R[I, J]$ is positive) design an efficient algorithm to determine whether or not there exists a sequence of distinct indices: $I_1, I_2, \dots, I_k$, where $1 \leq k \leq N$, such that $ R[I_1, I_2] \times R[I_2, I_3] \times \cdots \ \times R[I_{k-1}, I_k] \times R[I_k, I_1] > 1$. State your algorithm precisely and analyze the running time of your algorithm. \school{[NCTU CSIE 96]}
\end{Exercise}
\begin{Answer}
\end{Answer}

\section{Spanning tree}
\begin{Exercise}
\begin{enumerate}
\item Consider the following variation of the Minimum Spanning Tree problem: Given a graph $G$ of $n$ vertices and $m$ edges AND a minimum spanning tree $T$ of graph $G$, we wish to add new edge e with weight $w_e$ to $G$ forming a new graph $G'$ and construct the new minimum spanning tree of the new graph $G'$. Give an algorithm which constructs the minimum spanning tree of $G'$ in $O(n)$ time. \school{[NCU CSIE 102]}
\item Suppose that a graph $G$ has a minimum spanning tree already computed. How quickly can the minimum spanning be updated if a new vertex and incident edges are added to $G$? Please justify your answer.\school{[NTUT CSIE 98]}
\end{enumerate}
\end{Exercise}
\begin{Answer}
\end{Answer}

\section{Matching}
\begin{Exercise}
Let $X = \{1, \dots, n\}$. For a subset of $X$, we say that it covers its elements. Given a set $\mathcal{S} = \{ S_1, S_2, \dots, S_m\}$ of $m$ subsets of $X$ such that $\cup_{i=1}^m S_i = X$, the set cover problem is to find the smallest subset $T$ of $S$ whose union is equal to $X$, that is, $\cup_{S_i \in T} S_i = X$. Suppose that each subset $S_i \in \mathcal{S}$ contains only two elements. Can the set cover problem then be solved in polynomial time? If yes, please also design a polynomial-time algorithm to solve this set cover problem and analyze its time complexity. \school{[NTHU CSIE 101]}
\end{Exercise}
\begin{Answer}
\end{Answer}

\section{Network flow}
\begin{Exercise}
The escape problem is defined as the following. An $n \times n$ grid is an undirected graph consisting of $n$ rows and $n$ columns of vertices. We denote the vertex in the $i$-th row and $j$-th column by $(i, j)$. All vertices in a grid have exactly four neighbors, except for the boundary vertices, which are the vertices $(i, j)$ for which $i = 1$, $i = n$, $j = 1$, or $j = n$. Given $m \leq n^2$ starting vertices in the grid, the escape problem is to determine whether or not there are $m$ vertex-disjoint paths from the starting vertices to any m different vertices on the boundary. Vertex-disjoint paths mean that each vertex can be used at most once in the escape. Show how to convert the escape problem into the maximum flow problem. It is enough to give the conversion procedure. It is not require to show the correctness of your procedure. \school{[NTU CSIE 100]}
\end{Exercise}
\begin{Answer}
\end{Answer}

\begin{Exercise}
Suppose we are to assign $n$ persons to $n$ jobs. Let $C_{ij}$ be the cost of assigning the $i$-th person to the $j$-th job. Use a greedy method approach to write an algorithm that finds an assignment that minimizes the total cost of assigning all $n$ persons to all $n$ jobs. Analyze your algorithm, and give the time complexity using order notation. \school{[FJU CSIE 91]}
\end{Exercise}
\begin{Answer}
\end{Answer}

\section{Others}
\begin{Exercise}
Suppose you are asked to assign direction for each edge in the graph to make it a digraph such that each vertex can connect to each other vertex by some directed graph (i.e. strongly connected). How do you know whether such strongly connected orientation exists for an undirected graph $G$ of $n$ vertices and $m$ edges. Explain your method and discuss its complexity. \school{[NCKU IM 99]}
\end{Exercise}
\begin{Answer}
\end{Answer}

\begin{Exercise}
A tournament $T = (V, E)$ is a simple digraph of $|V| = n$ vertices and $|E| = \frac{n(n-1)}{2}$ edges, suppose you already know $\text{outdeg}[i]$, the outdegree for each vertex $i$. A tournament is transitive, whenever edge $(u, v) \in E$ and $(v, w) \in E$ implies $(u, w) \in E$. In other words, if there exists any $3$ vertices $i$, $j$, $k$ in $T$ with edges $(i, j)$, $(j, k)$, and $(k, i)$, then $T$ is NOT transitive. Now you want to check whether $T$ is transitive or not. \school{[NCKU CSIE 100]}
\end{Exercise}
\begin{Answer}
\end{Answer}


\begin{Exercise}
Given an undirected graph $G = (V, E)$ with $n = |V|$ vertices, four vertices of $G$, say, $u$, $v$, $x$, and $y$, are said to form a $4$-cycle if $(u, v)$, $(v, x)$, $(x, y)$ and $(y, u)$ are in $E$. Consider the problem of determining whether $G$ contains a $4$-cycle. A naive method by checking all possible $4$-combinations of the vertex set will need $\Omega(n^4)$ time to complete the job. Design a more efficient algorithm (i.e., the time complexity of your algorithm should be $O(n^k)$ with $k < 4$) to solve the problem. Analysis the execution time of your algorithm. \school{[NCU CSIE 95]}
\end{Exercise}
\begin{Answer}
\end{Answer}


\printbibliography[heading=subbibliography]
\end{refsection}
