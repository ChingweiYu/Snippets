\chapter{Analysis of algorithms}

\lettrine{T}{he} analysis of algorithms is to determine the amount of resource (time/space) necessary to execute algorithms.
By analyzing the resources used in algorithms, we can compare different algorithms theoretically.

The amount of resource used in an algorithm is usually represented by a function $T(n)$, where $n$ is the length of the input.
The goal of the analysis of algorithms is to find out the asymptotic growth rate of $T(n)$ in terms of $n$.
In computer science, since $n$ denotes the length of the input, we only care about positive values of $n$.
Similarly, since $T(n)$ denotes the amount of resource, we usually assume that $T(n)$ is positive.
This is not a real limitation, but can simplify the discussion.

Each textbook usually discusses the analysis of algorithms in the first several chapters.
For those who want to know further about the analysis of algorithms, you can watch the videos of \href{https://www.coursera.org/course/aofa}{Analysis of Algorithms} on Coursera and read books~\cite{Graham1994,Purdom2004,Sedgewick2013}.

\section{Recurrence relations}
During the exam, we are often given a recurrence relation $T(n) = a(n)T(b(n)) + f(n)$, and we need to give to a tight bound of $T(n)$. If the problem statements do not explicitly specify the base care, we usually assume that $T(n) = \Theta(1)$ when $n$ is small (less than some constant). In general, we can just focus on some particular values of $n$ as long as these values approach infinity\footnote{There should be a condition specifying what assumptions are applicable.}. For example, we can only consider $n = 2^i$ for all positive integer $i$. 
Making assumptions can make the analysis easier.

\begin{Exercise}[origin={YZU CSIE 90}]
\Question Prove or disprove $n^{2 + \frac{\sin n}{\lg n}} = O(n^2)$.
\Question If $T(n) = 49T(n/7) + n^2 \cos (\sqrt{n})$, please find the tight asymptotic order of $T(n)$.
\end{Exercise}
\begin{Answer}
According to the definition of big O notation, we say $f(n) = O(g(n))$ if there exists constants $c$ and $n'$, such that $f(n) \leq cg(n)$ for all $n > n'$.

For the first problem, we know that the $\sin$ function will fluctuate between $1$ and $-1$. Although $\frac{\sin n}{\lg n}$ will become very small when $n$ approaches infinity, we still cannot get constants $c$ and $n'$ such that $T(n) \leq cn^2$ for all $n > n'$.

For the second problem, we know that the $\cos$ function will fluctuate between $1$ and $-1$. So, the equation $n^2 \cos (\sqrt{n}) = O(n^2)$ holds, and we can apply the master theorem to get $T(n) = O(n^2 \lg n)$.
\end{Answer}

\subsection{Master theorem}
The most powerful technique in solving divide-and-conquer recurrence relation is the master theorem.
Several forms of the master theorem have been proven. Verma gives the following master theorem~\cite{Verma1994}:
\begin{theorem}
Let $T(n) = aT(n / b) + f(n)$ for all $n > 1$ and $T(1) = c$ for some constants $a \geq 1$, $b > 1$, $c > 0$, and non-negative function $f(n)$.
\begin{enumerate}
\item if $f(n) = O(n^{\lg_b a} / \lg n)(1 + \epsilon)$ for some constant $\epsilon > 0$, then $T(n) = \Theta(n^{\lg_b a})$.
\item if $f(n) = \Theta(n^{\lg_b a} \lg^k n)$ for some $k \geq 0$, then $T(n) = \Theta(n^{\lg_b a} \lg^{k+1} n)$.
\item if $f(n) = \Omega(n^{\epsilon + \lg_b a})$ for some constant $\epsilon > 0$, and if $af(n/b) \leq kf(n)$ for some constant $k < 1$ and all sufficiently large $n$, then $T(n) = \Omega(f(n))$.
\item if $f(n) = \Theta(n^{\lg_b a} / \lg n)$, then $T(n) = \Theta(f(n) \lg n \lg \lg n)$.
\end{enumerate}
\end{theorem}

\begin{remark} When you apply the master theorem, please pay attention to the following:
\begin{enumerate}
\item When the recursion involves floor or ceiling function, the master theorem does not apply. For example, $T(n) = T(\lfloor \frac{n}{2} \rfloor) + T(\lceil \frac{n}{2} \rceil) + n$. 
\item In order to apply the case 2 (the extended master theorem), $k$ must be non-negative. For example, case 2 does not apply in the case of $T(n) = 2T(\frac{n}{2}) + n/ \lg n$.
\item In order to apply the case 3, the regularity condition must be satisfied. For example, case 3 does not apply in the case of $T(n) = T(\frac{n}{2}) + n(2 - \cos n)$.
\end{enumerate}
\end{remark}

\begin{Exercise}[origin={NCTU CSIE 104}]
Let $T(n) = \Theta(f(n))$. Assume that $T(n)$ is a constant for sufficiently small $n$. Derive $f(n)$ in the simplest formula for each of the following $T(n)$.
\Question $T(n) = 2T(\frac{n}{2}) + \frac{n}{\lg^2 n}$. \label{master:1}
\end{Exercise}
\begin{Answer}

\paragraph{Problem~\ref{master:1}} By case 1, we know $T(n) = \Theta(n)$.

\paragraph{Problem~\ref{master:1} alternative solution} Suppose that $n = 2^k$. We have
\begin{align*}
T(n) &= 2T(\frac{n}{2}) + \frac{n}{\lg^2 n} \\
\equiv \quad T(n) &= 4T(\frac{n}{4}) + \frac{n}{\lg^2 n} + \frac{n}{(\lg (n)-1)^2} \\
\equiv \quad T(n) &= 2^k T(1) + n \sum_{i=}^k i^{-2} \\
\equiv \quad T(n) &= n + n \sum_{i=}^k i^{-2} \\
\end{align*}
Since $\sum_{i=}^k i^{-2}$ is lower bounded by one and is upper bounded by $\sum_{i=}^{\infty} i^{-2} = \zeta(2) = \frac{\pi^2}{6}$\footnote{This is the solution for the \href{https://en.wikipedia.org/wiki/Basel_problem}{Basel problem}.}, we have $T(n) = \Theta(n)$.
\end{Answer}

\subsection{Akra\--Bazzi method}
The Akra and Bazzi method provides a more general way to solve divide-and-conquer recurrence relations~\cite{Akra1998}. The following version is from~\cite{Leighton1996}.
\begin{theorem}
Let \[T(n) = 
\begin{cases}
\Theta(1)& 1 \leq n \leq n_0 \\
\sum_{i=1}^k a_iT(b_in +h_i(n)) + f(n) &\forall n > n_0
\end{cases}\]
where
\begin{enumerate}
\item $k \geq 1$ is a constant and for all $i$, $a_i > 0$ and $b_i \in (0, 1)$ are constants.
\item $\abs{f(n)}$ is polynomially-bounded.
\item for all $i$, $\abs{h_i(n)} = O(x /\lg^2x)$.
\item $n_0$ is large enough.
\end{enumerate}
Let $p$ be the unique solution for $\sum_{i=1}^k a_ib_i^p = 1$. Then
\begin{enumerate}
\item if $f(n) = O(n^{p-\epsilon})$ for some constant $\epsilon > 0$, then $T(n) = \Theta(n^p)$.
\item if $f(n) = \Theta(n^{p})$ for some constant $\epsilon > 0$, then $T(n) = \Theta(n^p \lg n)$.
\item if $f(n) = \Omega(n^{p+\epsilon})$ and $f(n)/x^{p+\epsilon}$ is non-decreasing for some constant $\epsilon > 0$, then $T(n) = \Theta(f(n))$.
\item $T(n) = \Theta\left(n^p\left(1 + \int_1^n \frac{f(u)}{u^{p+1}}du\right)\right)$.
\end{enumerate}
\end{theorem}

\begin{remark}
This version of the Akra\--Bazzi method can deal with floor and ceil function by picking $h_i$.
\end{remark}

\paragraph{General version}
Some of the requirements can be relaxed~\cite{Leighton1996}.
\begin{enumerate}
\item The second condition is called {\em polynomial-growth condition} and can be replaced by the following weaker requirement: $g(n)$ is nonnegative and exist constants $c_1$ and $c_2$ such that for all $i$ and for all $u \in [b_in + h_i(n), n]$, we have $c_1f(n) \leq f(u) \leq c_2 f(n)$.
\item The third condition can be replaced by the following weaker requirement: there exists a constant $\epsilon > 0$, $\abs{h_i(n)} \leq n / (\lg^{1 + \epsilon} n)$ for all $i$ and $n \geq n_0$.
\item The fourth condition is pretty technical and you can find the complete version in the original paper.
\end{enumerate}
Drmota and Szpankowski prove a more general theorem that can deal with floor and ceil functions directly~\cite{Drmota2013}.


\begin{Exercise}
Let $T(n) = \Theta(f(n))$. Assume that $T(n)$ is a constant for sufficiently small $n$. Derive $f(n)$ in the simplest formula for each of the following $T(n)$.
\Question $T(n) = 5T(\frac{n}{5}) + n/\lg n$. \label{ABmethod:1} \school{[NCTU CSIE 93]}
\Question $T(n) = T(\frac{n}{2} + \sqrt{n}) + n$. \label{ABmethod:2} \school{[NTU CSIE 103]}
\Question $T(n) = 4T(\frac{n}{5}) + T(\frac{n}{4}) + n$. \label{ABmethod:3} \school{[NCTU BIOINFO 93]}
\end{Exercise}
\begin{Answer}
\paragraph{Problem~\ref{ABmethod:1}} Suppose that $n = 5^k$. We have
\begin{align*}
 T(n)  & = 5(5T(\frac{n}{25}) + n/(5(\lg_5 n - \lg_5 5))) + n/\lg n \\
\equiv \quad T(n) & = 5^kT(1) + n/(\lg n + \lg (n-1) + \cdots + 1) \\
\equiv  \quad T(n) &=  \Theta(n \lg \lg n) \\
\end{align*}

\paragraph{Problem~\ref{ABmethod:1} another solution} apply the Akra\--Bazzi method. Set $k = 1$, $a_1 = 5$, $b_1 = 1/5$, and solve $p = 1$. We get
\[ T(n) = \Theta( n (1 + \int_{1}^n (x / \lg x) x^{-2} dx)) = \Theta(n \lg \lg n). \]
\begin{remark} Similarly, for any constant $c$, we have $T(n) = cT(\frac{n}{c}) + n/\lg n = \Theta(n \lg \lg n)$.
\end{remark}

\paragraph{Problem~\ref{ABmethod:2}} apply the Akra\--Bazzi method. Set $k = 1$, $a_1 = 1$, $b_1 = 1/2$, and solve $p = 0$. Then, we get
\[ T(n) = \Theta( 1 (1 + \int_{1}^n (x / x) dx)) = \Theta(n). \]

\paragraph{Problem~\ref{ABmethod:3}} apply the Akra\--Bazzi method. Set $k = 2$, $a_1 = 4$, $b_1 = 1/5$, $a_2 = 1$, $b_2 = 1/4$, and solve $p \approx 1.03$. Then, we get
\[ T(n) = \Theta( n^{p}), \text{where} p \approx 1.03. \]

\end{Answer}

\begin{Exercise}[origin={NCTU CSIE 92}]
Given positive constants $c'$, $c_1, c_2, \dots, c_k$, assume that $T(n) \leq  c'n + \sum_{i=1}^k T(c_in)$ and $\sum_{i=1}^k c_i < 1$. Prove $T(n) = O(n)$.
\end{Exercise}
\begin{Answer}
Apply the Akra\--Bazzi method.
\end{Answer}


\subsection{Full-history recurrence}
\begin{Exercise}[origin={NTU CSIE 90}]
Let $T(n) = \Theta(f(n))$. Assume that $T(n)$ is a constant for sufficiently small $n$. Derive $f(n)$ in the simplest formula for each of the following $T(n)$.
\Question $T(n) = n + \frac{4}{n} \sum_{i=1}^{n-1} T(i)$. \label{full-history:1}
\end{Exercise}
\begin{Answer}

\paragraph{Problem~\ref{full-history:1}}
\begin{align*}
T(n) & = n + \frac{4}{n} \sum_{i=1}^{n-1} T(i) \\
\equiv \quad nT(n) &= n^2 + 4\sum_{i=1}^{n-1} T(i) \\
\equiv \quad (n+1)T(n+1) &= n^2 + 4\sum_{i=1}^{n} T(i) \\
\equiv \quad (n+1)T(n+1) - nT(n) &= 2n + 1 + 4T(n) \\
\equiv \quad (n+1)T(n+1) &= (n+4)T(n) + 2n + 1 \\
\equiv \quad \frac{T(n+1)}{(n+2)(n+3)(n+4)} &= \frac{T(n)}{(n+1)(n+2)(n+3)} + \frac{2n+1}{(n+1)(n+2)(n+3)(n+4)} \\
\end{align*}

Let $S(n) = \frac{T(n)}{((n+1)(n+2)(n+3))}$.
\begin{align*}
S(n+1) & = S(n) + \frac{2n+1}{(n+1)(n+2)(n+3)(n+4)} \\
\equiv \quad S(n) &= \sum_{i=0}^{n-1} \frac{2i+1}{(i+1)(i+2)(i+3)(i+4)}\\
\equiv \quad S(n) &= \Theta(1) \\
\equiv \quad T(n) &= (n+1)(n+2)(n+3)S(n) \\
\equiv \quad T(n) &= \Theta(n^3) \\
\end{align*}
\end{Answer}


\subsection{Range transformation}
\begin{Exercise}[origin={NCTU CSIE 93}]
Let $T(n) = \Theta(f(n))$. Assume that $T(n)$ is a constant for sufficiently small $n$. Derive $f(n)$ in the simplest formula for each of the following $T(n)$.
\Question $T(n) = \sqrt{n} T(\sqrt{n}) + \sqrt{n}$. \label{range-transformation:1}
\end{Exercise}
\begin{Answer}
\paragraph{Problem~\ref{range-transformation:1}}
The trick is to divide $n$ by both side and then expand.
$\frac{T(n)}{n} =  \frac{T(\sqrt{n})}{\sqrt{n}} + n^{-0.5}$.
Suppose that $n = 2^{2^k}$. Let $S(k) = T(2^{2^k})/2^{2^k}$. 
\begin{align*}
T(n) & = S(k) = S(k-1) + 2^{-2^{k-1}} \\
\equiv \quad S(k) &= \Theta(1) + \sum_{i=1}^{k-1} 2^{-2^{k-1}} = \Theta(1) \\
\equiv \quad T(n) &= n \cdot S(k) = n \cdot \Theta(1) = \Theta(n) \\
\end{align*}
\end{Answer}

\subsection{Recursion trees}
\begin{Exercise}
Let $T(n) = \Theta(f(n))$. Assume that $T(n)$ is a constant for sufficiently small $n$. Derive $f(n)$ in the simplest formula for each of the following $T(n)$.
\Question $T(n) = 2T(\sqrt{n}) + \lg n$. \label{recursion-trees:1} \school{[NCKU CSIE 95]}
\Question $T(n) = nT(\sqrt{n}) + n^2 \lg n$. \label{recursion-trees:2} \school{[NTU CSIE 98]}
\end{Exercise}
\begin{Answer}

\paragraph{Problem~\ref{recursion-trees:1}} Suppose that $n = 2^{2^k}$. We have
\begin{align*}
T(n) &= 2^2T(\sqrt[4]{n}) + 2(\lg n - 1) + \lg n \\
\equiv \quad T(n) &= 2^k T(1) + \lg n + 2(\lg n)/2 + 4(\lg n)/4 + ... + 2^k(\lg n)/2^k \\
\equiv \quad T(n) &= 2^k T(1) + \lg n \sum_{i=1}^k 1 \\
\equiv \quad T(n) &= \Theta(\lg n \lg \lg n) \\
\end{align*}

\paragraph{Problem~\ref{recursion-trees:2}} Suppose that $n = 2^{2^k}$. We have
\begin{align*}
T(n) &= n^{1.5}T(\sqrt[4]{n}) + \frac{n^2 \lg n}{2} + n^2 \lg n \\
\equiv \quad T(n) &= n^2 T(1) + n^2 \lg n \sum_{i=0}^k 2^{-i} \\
\equiv \quad T(n) &= \Theta(n^2 \lg n) \\
\end{align*}

\end{Answer}

\subsection{Comparison}
\begin{Exercise}[origin={NCTU BIOINFO 93},difficulty=1]
Let $T(n) = \Theta(f(n))$. Assume that $T(n)$ is a constant for sufficiently small $n$. Derive $f(n)$ in the simplest formula for each of the following $T(n)$.
\Question $T(n) = T(\frac{n}{2}) + T(\sqrt{n}) + n$. \label{comparison:1} \school{[NCTU BIOINFO 93]}
\end{Exercise}
\begin{Answer}

\paragraph{Problem~\ref{comparison:1}}
By the recurrence relation, we have $T(n) = \Omega(n)$. Let $F(n) = F(n/2) + F(n/3) + n$. Since $n/3 >  \sqrt{n}$ for all $n > 9$, we have $F(n) \geq T(n)$.
By using the Akra\--Bazzi method, we get $F(n) = \Theta(n)$.
Thus, we have $T(n) = \Theta(n)$.
\end{Answer}

\printbibliography[heading=subbibliography]
